% !TeX encoding = UTF-8
% !TeX spellcheck = en_US

\documentclass[
	accentcolor=tud1a %Pick the color
]{tudreport}

% A couple of packages that may be useful
\usepackage{amsmath}
\usepackage{amsfonts}
\usepackage{amsthm}
\usepackage{algorithm2e}
\usepackage{listings}
\usepackage{xcolor}
\usepackage{tikz}
\usepackage{booktabs}
\usepackage{subfigure}
\usepackage[english]{babel}
\usepackage{blindtext}

% Defining JavaScript listings
\definecolor{mygreen}{rgb}{0,0.3,0}
\definecolor{myyellow}{rgb}{0.3,0.3,0}
\definecolor{mygrey}{rgb}{0.9,0.9,0.9}

\lstdefinelanguage{JavaScript}{
  keywords={break, case, catch, continue, debugger, default, delete, do, else, false, finally, for, function, if, in, instanceof, new, null, return, switch, this, throw, true, try, typeof, var, void, while, with},
  morecomment=[l]{//},
  morecomment=[s]{/*}{*/},
  morestring=[b]',
  morestring=[b]",
  ndkeywords={class, export, boolean, throw, implements, import, this},
  keywordstyle=\color{blue}\bfseries,
  ndkeywordstyle=\color{darkgray}\bfseries,
  identifierstyle=\color{black},
  commentstyle=\color{purple}\ttfamily,
  stringstyle=\color{red}\ttfamily,
  sensitive=true
}

\lstset{
    frameround=fttt,
    language=JavaScript,
    numbers=left,
    breaklines=true,
    breakatwhitespace=true,
    keywordstyle=\color{blue}\bfseries, 
    basicstyle=\ttfamily\color{black},
    numberstyle=\color{black},
    stringstyle=\color{myyellow},
    commentstyle=\color{mygreen},
    backgroundcolor=\color{mygrey}
    }
\lstMakeShortInline[columns=fixed]|

%\begin{lstlisting}[float,caption=Code Example,label=l:code_example]
%var a;
%console.log(a);
%if(a) {
%	// This is a comment
%	console.log('Yeah');
%}
%\end{lstlisting}

\begin{document}
\title{ARSE: Agile Responsive Simple Environment}
\subtitle{Team Echo: Project Report for the course Internet Praktikum TK in WS 2015/16}
\subsubtitle{Elmi Faisal Ali \\
Enkhtuul Gankhuyag \\
Floriment Klinaku \\
Masaud Yakubu Alhassan \\
Omar Antonio Erminy Ugueto \\
Satia Herfert
}

\maketitle

\tableofcontents

\chapter{Introduction}
\label{ch:introduction}

\blindtext

\chapter{Architecture}
\label{ch:architecture}

\chapter{User Documentation}
\label{ch:use-documentation}

In this chapter we will roughly explain how to use our application. For each state or thematic group of our application, there is a section that explains the graphical layout and the possible ways to interact with the application.

%TODO include the last tasks (like chat) somewhere.

\section{Profile Management}
\label{sec:profile-mgmt}

Explain: Register, Login, Logout, change Profile

\section{Project Management}
\label{sec:project-mgmt}

Explain: List of projects, create project, modify/delete project.

Explain: Project-navbar

\subsection{Product Backlog}
\label{sec:backlog}

Explain: Adding items, viewing/modifying/deleting items, reordering items, dragging sprint delimiter, starting sprint.

\subsection{Sprint Board}
\label{sec:sprint-board}

Explain: Cancelling sprint, closing sprint, dragging stories, viewing stories, assigning a team member, adding tasks, changing tasks, dragging tasks, collapse/expand to view tasks.

\subsection{User Management}
\label{sec:user-mgmt}

Explain: List of members, assign role, remove members, add members with a certain role

\subsection{Project Configuration}
\label{sec:proj-config}

Explain: Adding story types, removing story types, adding statuses, removing statuses

\chapter{Done Product Backlog Items}
\label{ch:done-pbis}

In the following we list all the items from the original Product Backlog that were developed in this project. Certain stories have been moved to make the actions available to a different class of users, so that the product is more compatible with the idea of Scrum. The roles in our project are users, registered users, team members (of a project), and product owners (of a project). This list also contains stories that were modified or added to the original backlog.

%TODO go through our modified backlog and see if we need to add more stories here.

\begin{itemize}
	\item As a user, I can
	\begin{itemize}
		\item register to the system with user name, email and password. (Unique usernames)
		\begin{itemize}
			\item Users can be assigned to different projects with different roles.
		\end{itemize}
		\item access the web application with different kinds of devices (mobile, tablet, pc ...).
	\end{itemize}
	
	\item As a registered user, I can
	\begin{itemize}
		\item edit my profile (name, password).
		\item see the list of projects I am a member of.
		\item create new projects (name and description), and become automatically the product owner of these projects.
	\end{itemize}

	\item  As a team member of a project, I can
	\begin{itemize}
		\item access the Product Backlog
		\item access the Sprint Board
		\item access past Sprints which includes its name, time interval, and story points burnt.
		\item copy backlog items to the Sprint backlog.
		\item create new tasks in the task board (a task always belongs to a user story).
		\item change tasks including their description as well as their status (New, in progress, done).
		\item assign tasks to a team member.
		\item add new product backlog items to the product backlog of a project
		\begin{itemize}
			\item product backlog items can have different statuses (new, in progress, done).
			\item product backlog items can be of different types, e.g. feature, enhancement, fix
		\end{itemize}
		\item change the content of user stories
		\item change the order (priority) of user stories in the product backlog
		%\item chat with my team members
		%\item access and search chat logs
	\end{itemize}
	
	\item As a product owner of a project, I can
	\begin{itemize}
		\item create a new Sprint (a new Sprint cannot be created if there's one already running) with name, description, time-box and initial Sprint Backlog.
		\item assign registered users to a project.
		\item assign roles to project members (team member, product owner).
		\item change the list of statuses assignable to product backlog items in the project settings.
		\item change the list of types assignable to product backlog items in the project settings.
		\item close a Sprint (all done stories will be removed).
		\item cancel a Sprint
	\end{itemize}
\end{itemize}

%TODO Include a list of undone stories (if we have any)
%Bonus features
%users must confirm their registration by clicking on a registration link sent via email
%feature of burndown charts
%use special interaction features with mobile devices like shaking

\bibliographystyle{abbrvnat}
\bibliography{references}
\end{document}
